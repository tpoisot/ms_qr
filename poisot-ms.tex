\documentclass[12pt,oneside]{article}

\usepackage[utf8]{inputenc}
\usepackage[T1]{fontenc}
\usepackage[letterpaper]{geometry}

\usepackage{authblk}
\renewcommand\Authfont{\small\scshape}
\renewcommand\Affilfont{\itshape\footnotesize}

\usepackage{lineno}

\usepackage{tikz, tkz-graph}
\usepackage{pgfplots}
\pgfplotsset{
	width = 7cm,
	height = 6cm,
    cycle list name = black white,
    tick align = outside
}

\usepackage{setspace}

\usepackage[doi=false,url=false,isbn=false,sorting=none,backend=bibtex]{biblatex}
\bibliography{/home/tpoisot/texmf/bibtex/bib/local/library.bib}

\title{An \emph{a posteriori} measure of network modularity}

\date{}

\author[1,2,*]{Timoth\'ee Poisot}

\affil[1]{D\'epartement de biologie, chimie et g\'eographie, Universit\'e du Qu\'ebec \`a Rimouski, 300 All\'ee des Ursulines, Rimouski (QC), G5L 3A1, Canada}
\affil[2]{Qu\'ebec Centre for Biodiversity Sciences, Stewart Biological Sciences Building, 1205 Dr.~Penfield Avenue, Montr\'eal (QC), H3A 1B1, Canada}
\affil[*]{email: \texttt{timothee.poisot@uqar.ca}}

\begin{document}

\maketitle

\begin{abstract} Measuring modularity is important to understand the structure
of networks, and has an important number of real-world implications. Several
measures exist to assess modularity, which when applied to the same network,
can return both different modularity values and different module compositions.
In this article, I propose an \emph{a posteriori} measure of modularity, which
is simply defined as the ratio of interactions between members of the same
modules \emph{vs.} members of different modules. I apply this measure to a
large dataset of 290 ecological networks representing host--parasite and
predator--prey interactions, to show how the results are easy to interpret,
and can reveal new features about modularity.
\end{abstract}

\doublespacing\linenumbers

\section{Introduction}

Modularity, the fact that groups of nodes within a network interact more
frequently with themselves than with other nodes, is an important property of
several systems, including genetic \parencite{espinosa-soto_specialization_2010,bauer-mehren_gene-disease_2011}, informatics
\parencite{fortuna_evolution_2011}, ecological
\parencite{olesen_modularity_2007}, and socio-economic
\parencite{saavedra_strong_2011} interactions, as well as biogeographic
patterns \cite{carstensen_biogeographical_2011,thebault_identifying_2011} and disease
spread management \cite{chades_general_2011}. Because of the relevance of
modularity for network properties, it is important to assess it correctly.
There exists several methods to measure network modularity, some of which rely
on the optimization of a given criterion
\cite{newman_modularity_2006,zhang_optimization_2008}, label propagation
\cite{barber_modularity_2007}, or combination of these approaches
\cite{liu_community_2010}. These methods return two elements. The first is a
value of modularity for the networks, most often within the 0\,--\,1 interval.
Each method has often a threshold value, above which a network is considered
to be modular. Increasing values reflect and increasingly modular structure.
The second element is a ``community partition'', \emph{i.e.} the attribution
of each node to a module.

Recently, \textcite{thebault_identifying_2012} showed that different measures
of modularity tailored to presence/absence matrices (\emph{i.e.} networks in
which links have no weight), gave roughly equal estimates of the significance
of modularity, but differed in the community partition they returned
(\emph{i.e.} the identity of nodes composing each module varied). In this
situation, one might look for a way to choose which community partition should
be used. As the criterion that is optimized by each method is different, one
possible way to compare the different community partitions is to use an
\emph{a posteriori} measure to quantify modularity, which can be applied to a
network regardless of the method used to obtain the community partition.

An important feature of modular networks is the occurrence of interactions
between nodes of different modules. They contribute to the propagation of
disturbances \parencite{olesen_modularity_2007}, flow of information
\cite{wiederhold_mediators_1992,leskovec_statistical_2008}, and cross-
regulation of biological processes \cite{hartwell_molecular_1999}, \emph{inter
alia} \cite{rosvall_maps_2008}. In addition to measuring how modular the
network is, determining to which extents modules are connected, and to
identify nodes and edges responsible for connecting modules, is thus a
valuable information. In this article, I propose an \emph{a posteriori}
measure of the proportion of interactions established between modules,
\emph{i.e.} edges connecting different communities. I apply this measure to
the community partition identified by the Louvain method on 290 ecological
networks, and show that it behaves in a similar way to other modularity
measures.

\section{The measure}

In this contribution I define the \emph{realized modularity}, termed $Q_R$.
$Q_R$ measures the extent to which edges, within a network, are established
between nodes belonging to the same module. For $E$ edges in a network, if $W$
of them are established between members of the same module, then

\begin{equation}
	Q_R = \frac{W}{E} .
 	\label{e:qr}
\end{equation} 

When there are no between-module links, then $W = E$ and $Q_R$ takes the
maximal value of 1. When between-module interactions are as numerous as
within-modules interactions, then $W = E/2$, and $Q_R$ takes the minimal value
of 1/2. To express the \emph{realized modularity} as a value between 0 and 1,
it is expressed as:

\begin{equation}
	Q_R' = 2\times Q_R - 1 .
	\label{e:tqr}
\end{equation}

The main advantage of $Q_R$ is that it is agnostic with regard to the measure
used to optimize modularity (and even to the method by which the nodes were
assigned to modules, which can be arbitrary), as it acts \emph{a posteriori},
\emph{i.e.} after nodes have been attributed to modules. It can therefore be
used to select the community detection method maximizing modularity. This
measure works on most type of networks, as it makes no difference if links are
directional, or if the networks is bipartite/unipartite. An illustration of
this measure is given in Figure~\ref{f:illu}.

\begin{figure}[tb]
	\begin{center}
		\definecolor{COLOR0}{rgb}{0.8941176470588236,0.17647058823529413,0.17647058823529413}
\definecolor{COLOR1}{rgb}{0.17647058823529413,0.8941176470588236,0.17647058823529413}
\definecolor{COLOR2}{rgb}{0.17647058823529413,0.17647058823529413,0.8941176470588236}
\definecolor{COLOR3}{rgb}{0.0,0.0,0.0}
\pgfdeclarelayer{background}
\pgfdeclarelayer{foreground}
\pgfsetlayers{background,main,foreground}
\begin{tikzpicture}[scale=0.2]
\node at (4.4348946,1.7752205) [circle, line width=1, fill=COLOR0,  inner sep=0pt, minimum size = 15pt,] (1) {};
\node at (0.5787612,0) [circle, line width=1, fill=COLOR0,  inner sep=0pt, minimum size = 15pt,] (2) {};
\node at (1.1898844,4.1412354) [circle, line width=1, fill=COLOR0,  inner sep=0pt, minimum size = 15pt,] (3) {};
\node at (-3.9965492,4.022908) [circle, line width=1, fill=COLOR0,  inner sep=0pt, minimum size = 15pt,] (4) {};
\node at (-1.46605,9.0563972) [circle, line width=1, fill=COLOR0,  inner sep=0pt, minimum size = 15pt,] (5) {};
\node at (5.0572659,9.5155794) [circle, line width=1, fill=COLOR0,  inner sep=0pt, minimum size = 15pt,] (6) {};
\node at (-8.8685646,13.9302869) [circle, line width=1, fill=COLOR1,  inner sep=0pt, minimum size = 15pt,] (7) {};
\node at (-13.0699387,10.7953665) [circle, line width=1, fill=COLOR1,  inner sep=0pt, minimum size = 15pt,] (8) {};
\node at (-13.9869095,16.1683201) [circle, line width=1, fill=COLOR1,  inner sep=0pt, minimum size = 15pt,] (9) {};
\node at (-17.889743,15.2090019) [circle, line width=1, fill=COLOR1,  inner sep=0pt, minimum size = 15pt,] (10) {};
\node at (-16.3414551,19.3981277) [circle, line width=1, fill=COLOR1,  inner sep=0pt, minimum size = 15pt,] (11) {};
\node at (-9.6989677,19.8633057) [circle, line width=1, fill=COLOR1,  inner sep=0pt, minimum size = 15pt,] (12) {};
\node at (12.1416794,16.3403877) [circle, line width=1, fill=COLOR2,  inner sep=0pt, minimum size = 15pt,] (13) {};
\node at (12.7044243,21.8733261) [circle, line width=1, fill=COLOR2,  inner sep=0pt, minimum size = 15pt,] (14) {};
\node at (7.2612694,17.2345871) [circle, line width=1, fill=COLOR2,  inner sep=0pt, minimum size = 15pt,] (15) {};
\node at (2.4059761,22.491687) [circle, line width=1, fill=COLOR2,  inner sep=0pt, minimum size = 15pt,] (16) {};
\node at (14.8854568,25.5035843) [circle, line width=1, fill=COLOR2,  inner sep=0pt, minimum size = 15pt,] (17) {};
\node at (16.6169861,21.3912292) [circle, line width=1, fill=COLOR2,  inner sep=0pt, minimum size = 15pt,] (18) {};
\node at (9.9692177,26.0839203) [circle, line width=1, fill=COLOR2,  inner sep=0pt, minimum size = 15pt,] (19) {};
\begin{pgfonlayer}{background}
\tikzset{EdgeStyle/.style = {->,shorten >=1pt,>=stealth, bend right=1}}
\tikzset{EdgeStyle/.style = {-, shorten >=1pt, >=stealth, bend right=1, line width=0.5, color=COLOR3}}
\Edge (1)(6)
\Edge (2)(1)
\Edge (2)(4)
\Edge (3)(1)
\Edge (3)(2)
\Edge (3)(6)
\Edge (4)(3)
\Edge (4)(5)
\Edge (4)(8)
\Edge (5)(3)
\Edge (5)(7)
\Edge (6)(5)
\Edge (6)(13)
\Edge (6)(15)
\Edge (8)(7)
\Edge (8)(9)
\Edge (9)(7)
\Edge (9)(10)
\Edge (9)(11)
\Edge (9)(12)
\Edge (10)(8)
\Edge (11)(10)
\Edge (12)(7)
\Edge (12)(11)
\Edge (13)(14)
\Edge (13)(18)
\Edge (14)(17)
\Edge (14)(18)
\Edge (15)(13)
\Edge (15)(14)
\Edge (16)(12)
\Edge (16)(15)
\Edge (17)(19)
\Edge (18)(17)
\Edge (19)(14)
\Edge (19)(16)
\end{pgfonlayer}
\end{tikzpicture}
	\end{center}
	\caption{A cartoon depiction of a modular network with links between modules. Nodes of the same modules are identified by different colors. This network has a modularity (Louvain method) of $Q = 0.527$. Out of the 36 interactions, 31 are established within modules, and 5 between modules. This gives a $Q_R$ value of 0.86, and $Q_R' = 0.72$.}
	\label{f:illu}
\end{figure}

A python implementation of this measure, using the \texttt{networkx} package,
is proposed at \texttt{https://gist.github.com/tpoisot/4947006}. It reads data in the edge list
format, and offers additional functions to generate null networks, as detailed
in the following section.

\section{Example application: realized modularity in ecological networks}

In this section, I analyze the modular structure of a large dataset of 290
ecological networks (187 food webs and 113 host-parasite networks) published in
previous meta-analyses \cite{gravel_trophic_2011,poisot_dissimilarity_2012}.
Modularity is an important feature of ecological interaction networks, which is
linked to their resilience
\cite{fortuna_nestedness_2010,stouffer_compartmentalization_2011}, stability
\parencite{thebault_identifying_2012}, biogeographic structure
\cite{flores_multi-scale_2012}, functioning \cite{thebault_food-web_2003}, and
to the evolutionary mechanisms involved in their assembly
\cite{flores_statistical_2011}. Notably, the occurrence of interactions between
and within modules plays a central role in the structure of pollination
networks \parencite{olesen_modularity_2007}, and help buffer the effect of
species extinctions \parencite{stouffer_compartmentalization_2011}.

\subsection{Data and analysis}

I used the Louvain method \cite{blondel_fast_2008} to detect modules, due to
its rapidity and efficiency on large networks. The Louvain method works in two
steps: first it optimizes modularity \emph{locally}, through clustering of
neighboring nodes. These clusters are, in the second steps, aggregated
together, until modularity ceases to increase. This method is known to give
values of modularity comparable to what is found using \emph{e.g.} simulated
annealing, and has been observed to give modules that have a functional
relevance \cite{blonder_fast_2000}.  Once the partition is returned by the
Louvain method, I recorded its realized modularity $Q_R'$, and its modularity
$Q$ (using the \textcite{newman_finding_2004} measure).

For each network, I compared the values of $Q$ and $Q_R'$ on the empirical
networks to their random estimate using a network null model. The null model is
defined as follows. For each node $n$ of the network, I measured its degree
$d_n$, its number of successors(the number of node it links to, or generality
in ecological terms, as \emph{per} \cite{schoener_food_1989}) $g_n$, and its
number of predecessors (the number of nodes that link to it, or vulnerability)
$v_n$. In each random network, for each pair of nodes $(i,j)$, the probability
that $i$ interacts with $j$ is given by

\begin{equation}
	P(i\rightarrow j) = \frac{1}{2}\left(\frac{g_i}{d_i}+\frac{v_j}{d_j}\right),
	\label{e:null}
\end{equation}

\noindent and conversely for $P(jrightarrow i)$. This null model allowed the generation
of pseudo-random networks through a Bernoulli process (in each replicate, the
occurrence of a link is randomly determined), with the same connectance, and
the same distribution of degrees, generality, and vulnerability, as the
original one (these properties are also conserved at the emph{node} level).
For each of the 289 networks, 1000 pseudo-random replicates are generated. For
each of them, the average value of $Q_R$ and $Q_R'$ are estimated along with
their 90\% confidence interval. When the empirical value lies outside the
confidence interval, it can be assumed that the modular structure of the network is
different than expected by chance.

\subsection{Results}

\begin{figure}[tbp]
	\begin{center}
		\begin{tikzpicture}
			\begin{axis}[
			xlabel = $Q$,
			ylabel = $Q_R'$,
			xmin = 0,
			xmax = 1]
				\draw[thick] (axis cs:0,0) to (axis cs:1,0);
				\addplot+[only marks] table[x=q,y=qr] {analyses/results.dat};
			\end{axis}
		\end{tikzpicture}		
	\end{center}
	\caption{Relationship between the modularity of the best partition using the Louvain method and the \emph{a posteriori} realized modularity. There exists a strong, positive relationship between the two variables. Worth noting is that fact that, for some networks, the best partition resulted in \emph{negative} versions of $Q_R'$, \emph{i.e.} there were more interactions between than within modules. Each dot correspond to a network.}
	\label{f:qqr}
\end{figure}

There is a strong, positive relationship, between the values of $Q_R'$ and the
values of $Q$ (Pearsons's product-moment correlation coefficient, as implemented in \texttt{R 2.15} \cite{team_r:_2008}, $\rho =
0.64$, 288 d.f., $p < 10^{-6}$), \emph{i.e.} networks for which a high
modularity is detected tend to have relatively few between-module links
(Figure~\ref{f:qqr}). It is worth noting that some $Q_R'$ values were negative:
in some cases, the best community division resulted in more interactions
between than within modules. This result highlights why using an \emph{a
posteriori} measure is useful: other measures of modularity do not reveal the
fact that there were more interactions between than within modules. $Q$ and
$Q_R'$ have different relationships with connectance (Figure~\ref{f:co}).
Increased connectance values resulted in lower modularity ($\rho = -0.61$, 288
d.f., $p < 10^{-6}$), but had no impact on $Q_R'$. This is a desirable
property, as it allows easy comparison with the $Q_R'$ values of networks with
extremely different connectances.

\begin{figure}[tbp]
	\begin{center}
		\begin{tikzpicture}
			\begin{axis}[
			xlabel = Connectance,
			ylabel = $Q$,
			ymax = 1,
			xmax = 0.5]
				\node at (rel axis cs:0.07,0.93) {\sffamily \textbf{A}};
				\addplot+[only marks] table[x=co,y=q] {analyses/results.dat};
			\end{axis}
		\end{tikzpicture}
		\begin{tikzpicture}
			\begin{axis}[
			xlabel = Connectance,
			ylabel = $Q_R'$,
			ymax = 1,
			xmax = 0.5]
				\node at (rel axis cs:0.07,0.93) {\sffamily \textbf{B}};
				\addplot+[only marks] table[x=co,y=qr] {analyses/results.dat};
			\end{axis}
		\end{tikzpicture}%		
	\end{center}
	\caption{Relationship between the two measures of modularity and network connectance. \textbf{A.} $Q$ is negatively affected by connectance, \emph{i.e.} densely connected networks are more likely not to be modular. \textbf{B.} $Q_R'$ is not affected by connectance, allowing to use it to compare different networks. Each dot correspond to a network.}
	\label{f:co}
\end{figure}

There is a linear relationship between the deviation from random expectation
of $Q$ and $Q_R'$ ($\rho = 0.78$, 288 d.f., $p < 10^{-6}$ --
Figure~\ref{f:dnull}). The deviations (respectively $\Delta Q$ and $\Delta
Q_R'$) are calculated as the empirical value, minus the average of the values
on the networks generated by the null model. As an example, a $\Delta Q$ less
than zero indicates that the empirical network is less modular than expected
by chance. Confidence intervals for the average of the null models were
typically very narrow (not represented in the figure to avoid cluttering --
see associated dataset), probably owing to the fact that the null model is
restrictive on the type of networks which are generated. It is worth noting
that for some networks, the diagnostic of the null model analysis conflicted.
It is then worth knowing why these networks are less modular but have fewer
links between modules, or conversely, than expected. This kind of analyses
will help further clarify the importance of species functional role on
ecological network modularity.

\begin{figure}[tbp]
	\begin{center}
		\begin{tikzpicture}
			\begin{axis}[
			xlabel = $\Delta Q$,
			ylabel = $\Delta Q_R'$,
			xmin = -0.25, xmax = 0.25,
			ymin = -0.6, ymax = 1.2]
				\draw[fill=red!20!white] (axis cs:-0.25,-0.6) rectangle (axis cs:0,0);
				\draw[fill=blue!20!white] (axis cs:0,0) rectangle (axis cs:0.25,1.2);
				\addplot+[only marks] table[x=dq,y=dqr] {analyses/results.dat};
			\end{axis}
		\end{tikzpicture}		
	\end{center}
	\caption{Linear relationship between the deviation from random expectation in $Q$ and $Q_R'$. Networks in the red area are detected as being less modular than expected both by $Q_R'$ and $Q$, while networks in the blue area are detected as being more modular. Although the agreement between the two measures is good (see main text for statistics), some networks are detected as having a higher than expected realized modularity $Q_R'$, despite a lower than expected modularity $Q$. Each dot correspond to a network.}
	\label{f:dnull}
\end{figure}

\section{Conclusions}

The $Q_R'$ measure presented here allows the estimation of the proportion of
interactions established between different modules in a network. This measure
can be analyzed much in the same way as other measures of modularity, but is
applied \emph{a posteriori}. As such, it can help choosing the ``best''
community partition according to the property of the network that one wants to
maximize. For example, choosing the partition giving the lowest $Q_R'$ can
help identify which species are more likely to act as connectors between
different modules.

\section{Grant information}

TP is funded by a PBEEE-FQRNT post-doctoral scholarship, and thanks the EEC
Canada Research Chair for providing computational support.

\textbf{Acknowledgements:} Thanks are due to the maintainers and contributors of
the free texttt{networkx}, texttt{scipy}, and texttt{numpy} packages used in
this project, and to Scott Chamberlain for discussions.

\printbibliography

\end{document}
